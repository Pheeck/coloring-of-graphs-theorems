\documentclass{article}

\usepackage[english]{babel}
\usepackage[utf8]{inputenc}
\usepackage{geometry}
\usepackage{csquotes}
\usepackage{amsthm}
\usepackage{amsmath}
\usepackage{amssymb}

% Table of contents
\usepackage{blindtext}
\usepackage{titlesec}
\usepackage{hyperref}
\hypersetup{
    colorlinks,
    citecolor=black,
    filecolor=black,
    linkcolor=black,
    urlcolor=black
}

\newtheorem*{theorem}{Theorem}
\newtheorem*{definition}{Definition}
\newtheorem*{lemma}{Lemma}
\newtheorem*{conjecture}{Conjecture}
\newtheorem*{observation}{Observation}
\newtheorem*{corollary}{Corollary}

\newcommand{\Z}{\mathbb{Z}}

\begin{document}

\section*{Note}
This document is a list of theorems and definitions taught in the 2023/24 run
of the Coloring of Graphs and Other Combinatorial Structures lecture. I also
added some notes concerning the material where it seemed appropriate to me.
Good luck with the exam. --Filip Kastl, \today

\tableofcontents

\newpage
\section*{Notation and some definitions}
\begin{itemize}
	\item Girth -- Length of shortest cycle
	\item $G^*$ -- Dual graph to $G$
	\item $G[A]$ -- Subgraph of $G$ induced by the vertex set $A$
	\item $\Delta(G)$ -- Max degree of graph $G$
	\item $G$ is $d$-degenerate -- Every subgraph has a vertex of degree
		$\le d$
	\item $||G||$ (in lecture notes) -- Number of edges of $G$
	\item $|G|$ (in lecture notes) -- Number of vertices of $G$
	\item Clique number $\omega$ of $G$ -- The size of the biggest clique
		in $G$
\end{itemize}

\section*{Good things to remember}
\begin{itemize}
	\item A graph is bipartite iff it doesn't contain any odd cycles
	\item $d$-degeneracy $\implies$ $(d+1)$-colorability
		\begin{itemize}
			\item
				Alg: Build a sequence of vertices so that every
				vertex has at most $d$ edges going left. Now
				color vertices greedily from left to right. To
				build the sequence always take the vertex with
				$\deg d$, put it on the leftmost position and
				remove it from the graph. Continue until the
				graph is empty.
		\end{itemize}
	\item Girth $\ge 4$ means that the graph is triangle-free
	\item In this lecture it suffices to talk about graph of girth $k, k
		\ge 3$ because cycles of length $2$ and less only exist in
		multigraphs.
	\item $2$-colorability is deciding if the graph is bipartite which can
		be done by a greedy algorithm in linear time
	\item Bridgeless is the same thing as $2$-edge-connected
\end{itemize}


\newpage
\section{List coloring} % Lecture 1
3.10.2023

\subsection*{Recap of theorems from other lectures}

\begin{theorem}[Brooks]
	Let $G$ be a connected graph, $G \not \simeq K_n, C_{2l+1}$. Then
	$\chi(G) \le \Delta(G)$.
\end{theorem}

\begin{theorem}[Vizing]
	Let $G$ be a simple graph (not a multigraph). Then $\Delta(G) \le
	\chi_e(G) \le \Delta(G)+1$
\end{theorem}

\begin{theorem}[Euler's formula]
	Let $G$ be a graph drawn on a surface of genus $g$. Then
	$$ |E(G)| \le |V(G)| + |F(G)| + g - 2 $$
	The equality holds for $2$-cell embeddings (each face homeomorphic to
	open disc).
\end{theorem}

\begin{corollary}
	For $|V(G)| \ge 3: |E(G)| \le 3|V(G)| + 3g - 6$
\end{corollary}

\begin{corollary}
	$${{2|E(G)|}\over{|V(G)|}} \le 6 + {{6(g-2)}\over{|V|}}$$
	(${{2|E(G)|}\over{|V(G)|}}$ is the average degree)
\end{corollary}

\begin{theorem}[Heawood's formula]
	Let $G$ be a graph drawn on a surface of genus $g$. Then
	$$\chi(G) \le
	\left \lfloor {{7 + \sqrt{24g+1}}\over{2}} \right \rfloor$$
\end{theorem}

\noindent
Deciding $3$-colorability of planar graphs is NP-complete.

\subsection*{List coloring}

\begin{definition}[List coloring]
	~
	\begin{itemize}
		\item \emph{$k$-list-assignment} is an assignment of lists of
			size $k$ for each vertex.

		\item A graph $G$ is \emph{$k$-list-colorable} (=$k$-choosable)
			if it can be colored by colors from any
			$k$-list-assignment $L$.

		\item The list chromatic number $\chi_l(G)$ is the min $k$ s.t.
			$G$ is $k$-list-colorable.
	\end{itemize}
\end{definition}

\noindent
For example $\chi_l(C_4) = 2$

\noindent
\textbf{Observations.}
\begin{itemize}
	\item $\chi(G) \le \chi_l(G)$
	\item If $G$ $d$-degenerate, $\chi_l(G) \le d + 1$
\end{itemize}

\begin{lemma}
	$\chi_l(K_{a,a^a}) = a + 1$

	(This is an example where $\chi_l$ and $\chi$ differ. $\chi(K_{a,a^a})
	= 2$)
\end{lemma}

\noindent
Exists $G$ planar s.t. $\chi_l(G) > 4$ (Voigt). Foreach $G$ planar $\chi_l(G)
\le 5$ (Thomassen). We are going to prove these claims in lecture 11.

\noindent
Can we extend $\chi$ theorems to $\chi_l$?
\begin{itemize}
	\item Heawood works for $\chi_l$ but only for $g > 0$
	\item Vizing's theorem for list coloring (edge list coloring i guess??)
		is an open problem
	\item Brook's theorem works for $\chi_l$
\end{itemize}

\begin{definition}[Degree list coloring]
	~
	\begin{itemize}
		\item \emph{degree-list-assignment} is a list-assignment s.t.
			$\forall v: |L(v)| \le \deg v$.

		\item A graph $G$ is \emph{degree-list-colorable}
			(=degree-choosable) if it can be colored by colors from
			any degree-list-assignment $L$.
	\end{itemize}
\end{definition}

\noindent
Could we strenghten Brook's $\chi_l$ theorem so that any graph $G \not \simeq
K_k, C_{2l+1}$ is degree-list-colorable? But this doesn't hold -- consider a
$k$-star graph.

\begin{definition}
	A connected graph $G$ is a \emph{Gallai tree} if each of its
	$2$-edge-connected component is $K_k$ or $C_{2l+1}$.
\end{definition}

\noindent
\textbf{Observation.} Gallai trees are not degree-list-colorable.

\begin{theorem}
	If $G$ connected, $G$ is not a Gallai tree, then $G$ is
	degree-list-colorable.
\end{theorem}

\begin{lemma}
	$G$ connected, $L$ degree-list-assignment, $\exists v: |L(v)| > \deg
	v$. Then $G$ is $L$-colorable.
\end{lemma}
\begin{proof}
	Take $v$ with $|L(v)| > \deg v$, color it with any color, remove the
	color from lists of its neighbors, remove $v$ from graph. Now the
	neighbors all have $|L(w)| > \deg w$. Repeat until everything is
	colored.
\end{proof}

\begin{lemma}
	$G$ connected, $L$ degree-list-assignment, $uv \in E(G)$, $v$ is not a
	cut vertex, $G$ is not $L$-colorable. Then $L(u) \subseteq L(v)$.
\end{lemma}

\begin{corollary}
	$G$ connected, $L$ degree-list-assignment, $uv \in E(G)$, $u, v$ not
	cut vertices, $G$ not $L-colorable$. Then $L(u) = L(v)$.
\end{corollary}

\begin{corollary}
	$G$ $2$-connected, $L$ degree-list-assignment, $G$ not $L$-colorable.
	Then $(\forall u, v \in V(G)) L(u) = L(v)$ and $G \simeq K_k$ or $G
	\simeq C_{2l+1}$.
\end{corollary}


\newpage
\section{Critical 1} % Lecture 2
10.10.2023

\noindent
In this lecture we define critical graphs and build a little theory around
them. Then using the theorem about degree-list-coloring from the previous
lecture we formulate a lower bound on the density of critical graphs. This
bound will be useful in the next lecture where we use it to build a polynomial
algorithm to decide colorability of many surface-drawn graphs.

\begin{definition}
	A graph $G$ is \emph{$c$-critical} if $\chi(G) = c$, but every proper
	subgraph $G$ is $(c - 1)$-colorable.
\end{definition}

\noindent
\textbf{Observations.}
\begin{itemize}
	\item $G$ is $2$-critical iff $G = K_2$
	\item $G$ is $3$-critical iff $G$ is odd cycle
	\item $G$ is $c$-colorable iff $G$ doesn't contain $(c+1)$-critical
		subgraph
	\item Each $(c+1)$-critical graph has minimum degree at least $c$
\end{itemize}

\begin{lemma}
	Let $G$ be a $(c+1)$-critical graph and let $S$ be the set of vertices
	of $G$ of degree $c$. Then each component of $G[S]$ is a Gallai tree.
\end{lemma}

\begin{lemma}
	If $F$ is a Gallai forest, $\Delta(F) \le c, F \not \simeq K_{c+1}, c
	\ge 3$, then
	$$ E(F) \le \left ({{c-1}\over2} + {1 \over c} \right ) V(F) $$
\end{lemma}

\begin{theorem}
	For $c \ge 3$ if $G$ is a $(c+1)$-critical graph and $G \not \simeq
	K_{c+1}$, then $G$ has average degree at least
	$$ c + {{c-2}\over{c^2 + 2c - 2}} $$
\end{theorem}

\noindent
Cases $c < 3$ are covered by observations we made. Also, consider graphs
$K_{c+1}$. Their average degree is $c$ so the theorem clearly cannot hold for
them. They are a class of critical graphs with the lowest average degree. They
certify that the lower bound on minimum degree we formulated as an observation
is tight.


\newpage
\section{Critical 2: Surfaces} % Lecture 3
17.10.2023

\noindent
In this lecture we present two algorithms able to color many types of
surface-embedded graphs in polynomial time (linear time in the number of
vertices with a constant factor polynomially dependent on the genus of the
surface). There are also two random results about triangulations and
quadrangulations thrown in.

\subsection*{First algorithm}

\begin{corollary}[...of Euler's formula]
	Let $G$ have girth $\ge k$ and $G$ not a forest, then
	$$ |E(G)| \le {k\over{k-2}}(|V(G)| + g - 2)$$
\end{corollary}

This theorem follows from the corollary:

\begin{theorem}
	Let $C_{k,g} = {{2k(g-2)}\over{k-2}}$ Let $G$ be a graph with girth $k
	\ge 3$ (no trees) that can be embedded in a surface of genus $g$. Then
	average degree of $G$ is at most $$ {{2k}\over{k-2}} +
	{C_{k,g}\over{|V(G)|}}$$
\end{theorem}

\noindent
This theorem is at the heart of the first algorithm. Let's fix $k,g$. Consider
the case where $|V(G)| > C_{k,g}$. This means that ${C_{k,g}\over{|V(G)|}} <
1$. Then the average degree is $< {{2k}\over{k-2}} + 1$ . For example if $k =
3$, the average degree is $< 7$. This means that the graph contains at least
one vertex of degree $6$.

This gives us a greedy algorithm similar to the one that is used to color a
$d$-degenerate graph. We find a vertex of degree $\le 6$ among uncolored
vertices, color it (there is at most $6$ colors among its neighbors, one color
must be left), repeat.

However a problem arises when we reach $|V(G)| \le C_{k,g}$. The graph may no
longer contain a vertex of degree $6$. But at this point the problem of
\enquote{is $G$ $7$-colorable} has been reduced to \enquote{is $G, |V(G)| \le
C_{k,g}$ $7$-colorable}. That problem can be solved by just comparing $G$ to
all $7$-colorable graphs of girth $\le k$ embeddable into a surface of genus
$g$.

The time complexity of this algorithm is $O(|V(G)| + |E(G)| + f(k, g))$ where
$f$ is a polynomial-bounded function. Given a graph of girth $k$ it can decide
its $c$-colorability, $c := {{2k}\over{k-2}} + 1$

\subsection*{Second algorithm}

From the previous theorem and the theorem at the end of the previous lecture,
we get this corollary (we combine lower and upper bound for average degree of a
critical graph)

\begin{corollary}
	Let $G$ be $(c + 1)$-critical, girth of $G$ is at least $k$, $c \ge
	\max \left ( 3, {{2k}\over{k-2}} \right )$ embedded in a surface of
	genus $g$, $G \not \simeq K_{c+1}$. Then $$ |V(G)| \le {{c^2 + 2c -
	2}\over{c - 2}} C_{k,g} $$ where $C_{k,g}$ is a term dependent only on
	$k$ and $g$.
\end{corollary}

This gives us a bound on the number of vertices of a $(c + 1)$-critical graph
of girth at least $k$. That means that the number of $(c + 1)$-critical graphs
with girth at least $k$ is also bounded.

This gives us another algorithm for checking $c$-colorability. Consider the
observation from the previous lecture that $G$ is $c$-colorable iff it doesn't
contain any $(c + 1)$-critical subgraphs. Also consider that if $G$ has girth
$\ge k$, its subgraphs have girth $\ge k$. So to check $c$-colorability of a
graph of girth $\le k$, we may just check compare its subgraphs with finitely
many $(c + 1)$-critical graphs of girth $\le k$. The algorithm runs in
polynomial time (there exists an algorithm that for a fixed graph $F$ and
surface $\Sigma$ decides $F \subseteq G$ in linear time).

This algorithm allows us to decide $c$-colorability for all the combinations of
$k$ and $c$ that the previous algorithm did and more. These are the $k,c$ pairs
that this algorithm doesn't work on:
\begin{itemize}
	\item $c = 3, k = 3$ -- This is NP-complete
	\item $c = 4, k = 3$ -- This is an open problem
	\item $c = 5, k = 3$ and $c = 3, k = 5$ -- These require a more
		complicated argument
	\item $c = 3, k = 4$ -- Here the number of critical graphs is not
		bounded but there still is a polynomial time algorithm
\end{itemize}

\subsection*{Triangulations and quadrangulations}
Some results specific to triangulations and quadrangulations. I am not sure why
we formulate these.

\begin{definition}
	\emph{Triangulation} is a graph embedded into a surface s.t. each face
	forms a $3$-cycle.

	\emph{Quadrangulation} is a graph embedded into a surface s.t. each
	face forms a $4$-cycle.
\end{definition}

\begin{lemma}[About triangulations]
	If $G$ is a triangulation of an orientable surface and only vertices
	$u,v$ have odd degree, then $u,v$ have the same color in any
	$4$-coloring of $G$.
\end{lemma}

\noindent
In particular if $u,v$ are adjacent, $G$ is not $4$-colorable. It was mentioned
that by connecting $u,v$, we can somehow construct infinitely many $4$-critical
graphs on torus. I don't know how, though.

\begin{lemma}[About quadrangulations]
	Suppose $G$ is a quadrangulation of the projective plane and $G$ is not
	bipartite. For every proper coloring, exists a face whose vertices have
	$4$-different colors.
\end{lemma}

\noindent
So $\chi(G) \ge 4$.


\newpage
\section{Critical 3: Kostochka and Yancey, Grötzsch} % Lecture 4
23.10.2023

\noindent
In this lecture we are going to prove prof. Dvořák's favourite theorem -- the
Grötzsch theorem -- using a claim about $4$-critical graphs.

\begin{observation}
	~
	\begin{itemize}
		\item $G$ $k$-critical $\Rightarrow$ $G$ $2$-connected
		\item $G$ $k$-critical $\nRightarrow$ $G$ $3$-connected
	\end{itemize}
\end{observation}

\noindent
Holds for both edge-connectivity and vertex-connectivity.

\begin{theorem}[Kostochka, Yancey]
	If $G$ is $(c+1)$-critical, $c \ge 3$, then the average degree is at
	least
	$$c + 1 - {1 \over c} - O \left ( {1 \over {|V(G)|}} \right )$$
\end{theorem}

\noindent
Without proof. The previous bound we had: $c + {{c-2}\over{c^2 + 2c - 2}} \sim
c + {1 \over c}$.

\begin{theorem}[Special case of Kostochka, Yancey]
	If $G$ is $4$-critical, then the average degree is at least
	$$ 3 + {1 \over 3} - {4 \over {3|V(G)|}} $$
	Equivalently
	$$ |E(G)| \ge {{5|V(G)| - 2} \over 3} $$
\end{theorem}

\begin{corollary}
	Let $G$ be a $4$-critical planar graph. Let $f_3 := $\# of its
	$3$-faces, $f_4 := $\# of its $4$-faces. Then
	$$ 2f_3 + f_4 \ge 8 $$
\end{corollary}

\noindent
From this corollary we prove Grötzsch's theorem.

\begin{theorem}[Grötzsch]
	Every triangle-free planar graph is 3-colorable.
\end{theorem}

\subsection*{Proving the special case of Kostochka, Yancey}

We will be proving the theorem reformulated using the potential $P(G) :=
5|V(G)| - 3|E(G)|$: If $G$ is $4$-critical, then $P(G) \le 2$. We'll be
considering the minimal counterexample: $G$ $4$-critical, $P(G) \ge 3$ with
min $|V(G)| + |E(G)|$. From now on, references to $G$ are to the minimal
counterexample.

\begin{lemma}
	$\forall S \subseteq V(G), |S| \ge 4, S \neq V(G):
	\exists G' \subseteq G: G[S] \subseteq G' \land P(G') \le P(G[S]) - 3$
\end{lemma}

\begin{lemma}
	If $u,v \in S, u \neq v$, both $u$ and $v$ have a neighbor outside of
	$S$ and $G[S] + uv$ is $3$-colorable, then either $G' \neq G$ or $P(G')
	\le P(G[S])-4$.
\end{lemma}

\begin{lemma}
	If $F \subseteq G, |V(F)| \ge 2$, then $P(F) \ge 6$.
\end{lemma}

\noindent
The proof continues in the next lecture.


\newpage
\section{Critical 4, Discharging 1} % Lecture 5
31.10.2023

\noindent
In this lecture we finish the proof of the special case of the Kostochka and
Yancey theorem. At the end of the proof we use a discharging argument. In the
last part of this lecture we introduce discharging properly.

\begin{lemma}
	$\forall H \subsetneq G, H \neq G - e \implies \forall u \neq v \in
	V(H): H + uv$ is $3$-colorable (by $G - e$ we mean a graph obtained by
	deleting any edge from $G$).
\end{lemma}

\begin{lemma}
	No triangle in $G$ contains two vertices of degree $3$.
\end{lemma}

\noindent
The following lemma is a strenghtening of the last lemma of the last lecture.

\begin{lemma}
	$H \subsetneq G, H \neq G - e, H \neq K_1, K_3 \implies P(H) \ge 7$
\end{lemma}

\begin{lemma}
	If $uv \in E(G), \deg u = \deg v = 3$ then each of $u,v$ is incident
	with a triangle.
\end{lemma}

\begin{corollary}
	Every vertex of $G$ of degree $3$ has $\le 1$ neighbor of degree $3$.
\end{corollary}

\noindent
The corollary follows from the seconds and fourth lemma of this lecture. The
corollary is all we need to prove the special case. It gives us a reducible
configuration of the minimal counterexample (a vertex of degree 3 with more
than one vertex of degree 3 as neighbor). To finish the proof of the special
case we use redistribution of charge.

\subsection*{Discharging arguments}
Here we introduce the discharging proof method. For the rest of this lecture
and in the next two lectures we prove claims using this method. The rest of
this lecture and the following lecture are both covered by the lecture notes
\enquote{Discharging and reducible configurations}.

\begin{theorem}
	Every planar graph without $4$-cycles is $4$-colorable.
\end{theorem}

\noindent
This of course is a consequence of the Four Color Theorem. However we prove
this statement without using 4CT to show off the discharging method.


\newpage
\section{Discharging 2} % Lecture 6
7.11.2023

\begin{lemma}[How to select initial charge?]
	Let $G$ be a planar graph. Let $a, b \in \mathbb{R}^+, a \le b$. Let
	$ch_0(v) = a \deg v  - 2b$ for $v \in V(G)$ and $ch_0(f) = (b - a)|f| -
	2b$ for $f \in F(G)$. Then the sum of all initial charges is $-4b < 0$.
\end{lemma}

\noindent
Typical initial charges are
$$a = 1, b = 3 \rightarrow ch_0(v) = \deg v  - 6, ch_0(f) = 2|f|-6$$
$$a = 1, b = 2 \rightarrow ch_0(v) = \deg v - 4, ch_0(f) = |f| - 4$$
All choices are in some sense equivalent. If you can complete a discharging
argument with $(a, b)$, you can do it with any $(a', b')$.

\begin{theorem}
	Every planar graph without $4$-cycles is $4$-choosable.
\end{theorem}

\noindent
This claim is very similar to the claim we proved in the last
lecture. However here we claim choosability instead of colorability.


\newpage
\section{Discharging 3: Grötzsch} % Lecture 7
14.11.2023

\noindent
In this lecture we try to prove Grötzsch's theorem using discharging. The
initial attempt to prove it directly fails so we attempt to prove a stronger
theorem (although with more preconditions) from which Grötzsch will follow.
We don't succeed in proving the first version of this stronger theorem. We do
succeed once we add more preconditions.

\begin{theorem}[Grötzsch]
	Every triangle-free planar graph is 3-colorable.
\end{theorem}

\begin{theorem}[Not true]
	Suppose $G$ is triangle-free planar graph with the other face bounded
	by a cycle $C, |C| \le 6$, $\psi$ is a $3$-coloring of $C$, then $\psi$
	extends to a $3$-coloring of $G$.
\end{theorem}

\begin{theorem}[True and implies Grötzsch]
	Suppose $G$ is triangle-free planar graph with the other face bounded
	by an \textbf{induced cycle} $C, |C| \le 6$, $\psi$ is a $3$-coloring
	of $C$, then $\psi$ extends to a $3$-coloring of $G$.
	\textbf{We also assume that if $|C| = 6$, $\psi$ is not a cyclic
	coloring.}
\end{theorem}


\newpage
\section{Nowhere-zero flows 1} % Lecture 8
21.11.2023

\noindent
In contrast to standard flows, $A$-flows aren't bound by edge capacities, don't
have source and target vertices and are based on finite abelian groups instead
of real numbers.

This lecture maps nicely to the \enquote{Nowhere-zero flows} lecture notes from
prof. Dvořák's website.

\subsection*{Notation}
For edge $e = uv$, $-e := vu$.

\subsection*{$A$-flows}

\begin{definition}[$A$-flow]
	Let $A$ be an abelian finite group. An \emph{A-flow} in a graph $G$ is
	a function $f: E(G) \rightarrow A$ such that
	\begin{itemize}
		\item $\forall uv \in E(G): f(uv) = -f(vu)$
		\item $\forall v \in V(G): \sum_{uv \in E(G)}{f(uv)} = 0$
			(Kirkhof)
	\end{itemize}
	For $S \subseteq V(G)$, let $f(S) = \sum_{u \in S, v \in V(G) \setminus
	S}{f(uv)}$
\end{definition}

\begin{definition}
	A \emph{nowhere-zero flow} is an $A$-flow such that $\forall e \in E:
	f(e) \neq 0$.
\end{definition}

\noindent
\textbf{Observations.}
\begin{itemize}
	\item $G$ has nowhere-zero $\Z_2$-flow iff $G$ is eulerian.
	\item $3$-regular $G$ has a nowhere-zero $\Z_3$-flow iff $G$
		bipartite.
	\item $3$-regular $G$ has a nowhere-zero $(\Z_2 \times \Z_2)$-flow iff
		$G$ is $3$-edge-colorable.
	\item For every $A$-flow: $\sum_{u \in S, v \in V(G) \setminus S}f(uv)
		= 0$
	\item $G$ has nowhere-zero $A$-flow $\implies \nexists$ edge-cut of
		size $1$ (bridgeless)
\end{itemize}

\subsection*{$A$-tensions}

\begin{definition}[$A$-tension]
	Let $A$ be an abelian finite group. An \emph{A-tension} in a graph $G$
	is a function $t: E(G) \rightarrow A$ such that
	\begin{itemize}
		\item $\forall uv \in E(G): t(e) = -t(-e)$
		\item $\forall$ cycle $C$ in $G: t(C) = 0$
	\end{itemize}
\end{definition}

\noindent
These lemmas will help us interchange $A$-colorings, $A$-flows and
$A$-tensions.

\begin{lemma}
	If $t$ is an $A$-tension in graph $G$ and $W$ a closed walk, then
	$t(W) = 0$.
\end{lemma}

\begin{lemma}
	Graph $G$ has a propper coloring using $A$ iff $G$ has a nowhere-zero
	$A$-tension.
\end{lemma}

\begin{lemma}
	A planar connected graph $G$ has a nowhere-zero $A$-tension iff $G^*$
	has nowhere-zero $A$-flow.
\end{lemma}

%\begin{lemma}
%	$t$ is an $A$-tension in a planar connected graph $G$ iff $t^*$ is an
%	$A$-flow in $G^*$.
%
%	(For edge $uv$ in $G$, we set $t^*(gh) := t(uv)$
%	where $g$ is the face on the right of $uv$ and $h$ is the face on the
%	left of $uv$.)
%\end{lemma}

\subsection*{Applying the $A$-flow theory}

\textbf{Observations.}
\begin{itemize}
	\item $G^*$ is $3$-regular iff $G$ is a triangulation
	\item Plane triangulation $G$ is $3$-colorable iff $G$ is eulerian
	\item Plane triangulation $G$ is $4$-colorable iff $G^*$ (a planar
		$3$-regular bridgeless graph) is $3$-edge-colorable
\end{itemize}

\noindent
The dual in the last observation is bridgeless because $G$ would have to be a
multigraph (have loops) in order for its dual to have bridges.

\begin{theorem}
	The following claims are equivalent:
	\begin{itemize}
		\item Every planar graphs is $4$-colorable
		\item Every planar $3$-regular bridgeless graph is
			$3$-edge-colorable
	\end{itemize}
\end{theorem}

\subsection*{What is known and not known}
These are open problems:

\begin{conjecture}[$5$-flow-conjecture]
	Every bridgeless graph has a nowhere-zero $\Z_5$-flow.
\end{conjecture}

\begin{conjecture}[$4$-flow-conjecture]
	Every bridgeless graph not containing the Petersen graph as a minor has
	a nowhere-zero $\Z_4$-flow (or $\Z_2 \times \Z_2$ -- it doesn't
	matter).
\end{conjecture}

\noindent
The $4$-flow-conjecture would imply the $4$-color theorem.

\begin{conjecture}[$3$-flow-conjecture]
	Every $4$-edge-connected graph has a nowhere-zero $\Z_3$-flow.
\end{conjecture}

\noindent
The $3$-flow-conjecture would imply the Grötzsch's theorem.

\noindent
What we do know:
\begin{itemize}
	\item $G$ is bridgeless $\implies$ nowhere-zero $\Z_6$-flow
	\item $G$ is $4$-edge-connected $\implies$ nowhere-zero $\Z_4$-flow
	\item $G$ is $6$-edge-connected $\implies$ nowhere-zero $\Z_3$-flow
\end{itemize}

\noindent
Actually, we are going to prove the second result and a weaker result for
bridgeless graphs in the next lecture.

\subsection*{Number of nowhere-zero $A$-flows}
$\chi^*(G, A) := \#$ of nowhere-zero $A$-flows in $G$

\noindent
\textbf{Observations.}
\begin{itemize}
	\item $E(G) = \emptyset \rightarrow \chi^*(G,A) = 1$
	\item $e \in E(G)$
		\begin{itemize}
			\item $e$ bridge $\implies \chi^*(G, A) = 0$
			\item $e$ loop $\implies \chi^*(G, A) =
				(|A|-1)\chi^*(G-e,A)$
			\item $e$ normal $\implies \chi^*(G, A) = \chi^*(G / e,
				A) - \chi^*(G - e, A)$
		\end{itemize}
	\item $|A_1| = |A_2| \implies \chi^*(G,A_1) = \chi^*(G,A_2)$
\end{itemize}


\newpage
\section{Nowhere-zero flows 2, Four Color Theorem 1} % Lecture 9
28.11.2023

\subsection*{Nowhere-zero flows}
Our goal in this lecture is to prove these two theorems.

\begin{theorem}
	If $G$ is $4$-edge-connected, then $G$ has a nowhere-zero
	$\Z_2^2$-flow.
\end{theorem}

\begin{theorem}
	If $G$ is $2$-edge-connected (=bridgeless), then $G$ has a nowhere-zero
	$\Z_2^3$-flow.
\end{theorem}

\begin{lemma}
	If $T$ is a spanning tree of a connected graph $G$, then $G$ has an
	$A$-flow which is zero only on a subset of edges of $T$.
\end{lemma}

\begin{theorem}[Nash-Williams]
	If $G$ is $2k$-edge-connected, then $G$ has $k$ pairwise disjoint
	spanning trees.
\end{theorem}

\noindent
Without proof.


\subsection*{Four Color Theorem}
In the rest of this lecture and in the following one we will go through a part
of the proof of the Four Color Theorem theorem to ilustrate the principle . The
whole proof would be too long and technical.

We will use the discharging technique. Thanks to an equivalence proven in the
previous lecture we can focus on proving this statement instead of proving the
Four Color Theorem directly.

\begin{theorem}[The Four Color Theorem, edge coloring version]
	Every planar bridgeless $3$-regular graph is $3$-edge-colorable.
\end{theorem}

\noindent
We are going to consider the minimal counterexample in the number of vertices.

We are going to use the notation $G_A$ meaning subgraph of $G$ induced by $A$
with edges leaving $A$ included and $S(G, A) := \{ f: \{1,...,k\} \rightarrow
\{1,2,3\}: \exists$ $3$-edge-coloring of $G_A$ s.t. $\forall i:$ the $i$-th
edge leaving $A$ has color $f(i) \}$, where $k$ is the number of edges leaving
$A$.

\begin{observation}
	~
	\begin{itemize}
		\item Given a cut $(A,B)$, $G$ is $3$-edge-colorable iff
			$S(G,A) \cap S(G,B) \neq \emptyset$
		\item If $\varphi \in S(G,A)$, then
			$|\varphi^{-1}(1)| \equiv |\varphi^{-1}(2)|
			\equiv |\varphi^{-1}(3)| \pmod{2}$
		\item If $G$ smallest counterexample, $G_B$ not a forest, then
			$S(G,A) \neq \emptyset$
	\end{itemize}
\end{observation}

\begin{definition}
	An edge cut $(A, B)$ is \emph{cyclic} if neither $G_A$ nor $G_B$ is a
	forest.
\end{definition}

\begin{lemma}
	A minimal counterexample is cyclically $4$-edge-connected.
\end{lemma}

\begin{definition}[Kempe consistency]
	Let $S$ be a set of functions $f: \{1,...,k\} \rightarrow \{1,2,3\}$.
	$S$ is \emph{kempe consistent} if
	$\forall \varphi \in S: \forall c_1 \neq c_2 \exists$ planar matching
	on $\varphi^{-1}(c_1) \cup \varphi^{-1}(c_2)$ s.t. $S$ contains all
	$\varphi'$ obtained from $\varphi$ by switching $c_1$ and $c_2$ on a
	subset of the matching.
\end{definition}

\noindent
Okay, but what does this mean? Let's start with \emph{planar matching}. Assume
we have an odd number of objects ordered in some way. Put them in order on the
rim of a disc. Now pair the objects. If the pairing allows you to draw lines
between objects of each pair without the lines intersecting, you have a planar
matching.

Now we have $k$ objects -- the $k$ edges leaving $A$. We also
have a set $S$ of colorings of these edges. Take a coloring $\varphi$
from the set. If the set $S$ is \emph{kempe consisten}, you should be able to
do the following and end up with a coloring $\varphi'$ also present in $S$.

Pick any two different colors $c_1,c_2 \in \{1,2,3\}$. Pick any planar matching
on edges leaving $A$ of colors $c_1$ and $c_2$. Now for any pairs, flip their
colors (a pair of edges both colored $c_1$ becomes colored $c_2$, a pair of
edges where the first one is colored $c_1$ and the second $c_2$ becomes a pair
where the first on is colored $c_2$ and the second $c_1$).

\begin{observation}
	$S(G, A)$ is Kempe consistent.
\end{observation}

\begin{lemma}
	A minimal counterexample is cyclically $5$-edge-connected.
\end{lemma}


\newpage
\section{Four Color Theorem 2} % Lecture 10
5.12.2023

\noindent
In this lecture we finish the excursion into the proof of the Four Color
Theorem.

\begin{lemma}
	A minimal counterexample is cyclically $6$-edge-connected except for
	cuts around $5$-cycles.
\end{lemma}

\noindent
There was a lemma about a graph consisting of four $5$-cycles not being a
subgraph of any minimal example. See someones notes for that. I don't want to
draw it in tex.

\begin{lemma}
	A minimal counterexample must contain a vertex of degree $\in \{7, ...,
	11\}$.
\end{lemma}


\newpage
\section{Results for list coloring} % Lecture 11
12.12.2023

\begin{theorem}[Voigt'94]
	There exists $G$ planar and $L$ assigment of size $4$ s.t. $G$ is
	not $L$-colorable.
\end{theorem}

\noindent
The proof is based on gadgets which we use to construct a planar graph and its
list assignments so that the graph is not list-colorable.

\begin{theorem}[Thomassen'93]
	Let $G$ be planar. $\chi_l(G) \le 5$.
\end{theorem}

\begin{theorem}[List coloring analog of Grötzsch doesn't hold]
	There exists $G$ planar triangle-free and $L$ assignment of size $3$
	s.t. $G$ is not $L$-colorable.
\end{theorem}

\noindent
We might think that Grötzsch's theorem might hold for list chromatic number
too -- for each $G$ planar $\chi_l \le 3$. This theorem provides a
counterexample to this claim. The proof is based on the same gadget technique
as Voigt'94.

However, these weaker claims hold:

\begin{observation}
	Let $G$ be planar triangle-free $\chi_l(G) \le 4$.
\end{observation}

\begin{theorem}
	Let $G$ be planar with girth $\ge 5$. $\chi_l(G) \le 3$.
\end{theorem}


\newpage
\section{Different coloring concepts} % Lecture 12
19.12.2023

\noindent
Taken from the \enquote{Circular coloring} lecture notes:

\noindent
Suppose we have a list of tasks, each taking one unit of time to accomplish.
The tasks can be performed in parallel, however some of the tasks conflict:
they cannot be worked on at the same time. This is expressed by a
\emph{conflict graph} G, whose vertices are the tasks and edges join pairs of
conflicting tasks. There are several scheduling problems one can consider in
this setting:

\begin{itemize}
	\item Suppose the tasks are indivisible—once you start working on a
		task, you must finish it (and it takes one unit of time, during
		which you cannot work on the conflicting tasks). Hence, during
		each unit of time, you can finish in parallel tasks
		corresponding to an independent set in G, and thus the minimum
		time needed to finish all tasks is exactly the chromatic number
		$\chi(G)$ of $G$.
	\item Suppose the tasks are divisible—you can interrupt your work on a
		task at any moment and continue later, and the task is finished
		once you devoted one unit of time to it in total. In this case,
		the minimum time needed to finish all tasks is exactly the
		fractional chromatic number $\chi_f(G)$ of $G$.
	\item The circular chromatic number of $G$ corresponds to the case
		that the tasks are indivisible, but they are to be performed
		repeatedly; we now ask to create a periodic schedule in which
		all tasks are performed for one consecutive unit of time every
		$t$ units of time, and we ask what is the minimum $t$ for that
		this is possible.
\end{itemize}

\begin{definition}[Circular coloring]
	~
	\begin{itemize}
		\item \emph{$r$-circular coloring} of a graph $G$ is a function
			$\varphi: V(G) \rightarrow$ open unit intervals of a
			circle of circumference $r$ s.t. $$ (\forall uv \in
			E(G)) \varphi(u) \cap \varphi(v) = \emptyset $$
		\item \emph{Circular chromatic number} $\chi_c(G)$ is defined
			as $\inf\{r \in \mathbb{R}:$ G has an $r$-circular
			coloring $\}$
	\end{itemize}
\end{definition}

\begin{definition}[Fractional coloring]
	~
	\begin{itemize}
		\item \emph{Fractional coloring} of a graph $G$ is a function
			$\varphi: V(G) \rightarrow$ measurable sets of measure
			$1$ s.t. $$ (\forall uv \in E(G)) \varphi(u) \cap
			\varphi(v) = \emptyset $$
		\item $|\varphi| :=$ measure of $\bigcup_{v \in V(G)}
			\varphi(v)$
		\item \emph{Fractional chromatic number} $\chi_f(G)$ is defined
			as $\inf\{|\varphi|: \varphi$ is a fractional coloring
			of $G\}$
	\end{itemize}
\end{definition}

\subsection*{Circular coloring}

\begin{observation}
	~
	\begin{itemize}
		\item $\chi_c + 1 > \chi \ge \chi_c \ge \chi_f \ge \omega$
		\item There's an $O(n!)$ LP-based alg for finding $\chi_c$ and
			$\chi_c$ is always rational.
	\end{itemize}
\end{observation}

\noindent
The consequence of the second observation is that we can write min instead of
inf in the definition of $\chi_c$.

\begin{theorem}
	If $\chi_c(G) \le {a \over b}, a,b \in \mathbb{N}$, then $G$ has an $a
	\over b$-circular coloring in which the intervals start at multiples of
	$1 \over b$.
\end{theorem}

\noindent
This effectively makes a discrete problem from a continuous one!

\begin{definition}[Discrete definition of circular coloring]
	\emph{$a \over b$-circular coloring} of a graph $G$ is a function
	$\varphi: V(G) \rightarrow \{0, ..., a-1\}$ s.t.
	$$(\forall uv \in E(G)) a - b \ge | \varphi(u) - \varphi(v) | \ge b $$
\end{definition}

\subsection*{Fractional coloring}

Suppose I have a schedule where interrupting tasks and working on them later is
acceptable. I can divide the schedule into slices based on points in time where
tasks begin or end. In each of this slice a certain independent set of the
conflict graph will be active.

Therefore I can view the problem of fractional coloring as the problem of
assigning durations to independent sets of the conflict graph.

\begin{definition}[Independent set definition of fractional coloring]
	~
	\begin{itemize}
		\item \emph{Fractional coloring} of graph $G$ is a function
			$\psi:$ independent sets of $G \rightarrow
			\mathbb{R}_0^+$ s.t.
			$$ \forall v \in V(G): \sum_{I \ni v} \psi(I) = 1 $$
		\item \emph{Fractional chromatic number} of $G$ is defined
			as
			$$ \chi_f(G) = \min \sum_I \psi(I) $$
	\end{itemize}
\end{definition}

\noindent
Note that this can be solved by linear programming so -- again -- $\chi_f$ is
always rational and we can write min instead of inf in its original definition.

The program has exponentially many variables though, so this does not give us a
polynomial algorithm.

\begin{definition}
	\emph{$(a:b)$-coloring} of a graph $G$ is a function $\varphi:
	\binom{\{1, ..., a\}}{b} \rightarrow V(G)$ s.t.
	$$ \forall uv \in E(G): \varphi(u) \cap \varphi(v) = \emptyset $$
\end{definition}

\begin{theorem}
	$\chi_f(G) = \min \{ {a \over b} : G$ has a $(a : b)$-coloring $\}$
\end{theorem}

\noindent
Not sure if this theorem was proven in the lecture or just stated.

\begin{definition}[Kneser graphs]
	$K_{a:b}$ is a graph with vertices $\binom{\{1,...,a\}}{b}$ s.t. $xy
	\in E(K_{a:b}) \Leftrightarrow x \cap y = \emptyset$
\end{definition}

\noindent
$K_{5:2}$ is the Peterson graph.

\begin{observation}
	$\chi_f(K_{a:b}) = {a \over b}$
\end{observation}

\noindent
However Kneser graphs have large chromatic numbers:

\begin{theorem}
	For $a \ge 2b: \chi(K_{a:b}) = a - 2b + 2$
\end{theorem}

\noindent
From this follows that $K_{10:4}$ doesn't have a $(5:2)$-coloring. Also, this

\noindent
Not sure if this theorem was proven in the lecture or just stated.
theorem rejects the idea that $\chi_f$ might be proportionate to $\chi$.


\newpage
\section{Counting method} % Lecture 13
9.1.2024

\begin{definition}
	An \emph{acyclic coloring} of a graph $G$ is a proper coloring s.t.
	each subgraph induced by any two colors is a forest.
\end{definition}

\noindent
Equivalently \enquote{there are at least $3$ colors on any cycle}

\begin{theorem}[Borodin]
	Let $G$ be planar. $\chi_a(G) \le 5$.
\end{theorem}

\noindent
Without proof.

\begin{observation}
	If $\chi_a(G) \le K$, then $|E(G)| \le (K-1)|V(G)|$
\end{observation}

\noindent
\enquote{If we want $\chi_a$ to be reasonably low we are looking for sparse
graphs.}

\begin{theorem}
	$\chi_a(G) \le \lceil \Delta + 3 \Delta^{3 \over 2} \rceil$
\end{theorem}

\noindent
So acyclic chromatic number can be bounded by a function of max degree.

\begin{theorem}
	For $\Delta \ge 10^{100}$, every triangle-free graph $G$ of max degree
	$\le \Delta$ has $\chi(G) \le \left \lceil {{4\Delta} \over
	{\log{\Delta}}} \right \rceil$.
\end{theorem}


\end{document}
